\documentclass[11pt]{article}
\usepackage[utf8]{inputenc}
\usepackage[spanish]{babel}
\usepackage{apacite}
\usepackage[left=2cm,top=2cm,right=2cm,bottom=2cm]{geometry}
\usepackage{natbib}
\usepackage{graphicx}
\usepackage{fancyhdr}

\title{Ensayo 2:\\ Navegación y manejo autónomos }
\author{Murrieta Villegas, Alfonso}



\graphicspath{ {./images/} }
\pagestyle{fancy}
\lhead{Facultad de ingeniería}
\rhead{UNAM}
\rfoot{Inteligencia Artificial}
\begin{document}

\maketitle

\section{Resumen}


Una de las mayores tendencias en el mercado por parte de empresas tan importantes como Tesla o ROS es la conducción o manejo automático y si bien, esta rama de la Inteligencia Artificial es de las más populares dentro del mundo de los automóviles, la realidad es que no es el único sector industrial donde se puede aplicar.\\

Sin duda dentro de la misma robótica como en casos concretos de robots de búsqueda o incluso de exploración, la navegación autónoma es una de las habilidades y capacidades mayormente implementadas debido a la amplia variedad de soluciones y tareas que los robots pueden hacer sin necesidad de estar completamente supervisados. \\

Realmente de poco sirve construir robots si estos nos son capaces de hacer nada, hoy en día, se ha logrado que los robots sean capaces de ejecutar tareas útiles para nosotros. Y para ello, debemos dotarlo de habilidades, por ejemplo y aunque suene muy mundano si queremos que un robot nos traiga un refresco del refrigerador, debemos asegurarnos de que éste posea las habilidades de navegar por nuestra casa, de manipular objetos con sus brazos y de agarrar objetos, entre otras cosas más.


\section{Navegación Autónoma en Robots}

Antes que nada, debemos entender por 'navegación autónoma' como la capacidad de ir de un punto del espacio a otro evitando obstáculos. Esta habilidad ha sido considerada como vital para los robots,ya que hoy en día no nos sirve un robot que no sea capaz de moverse en su entorno. Incluso históricamente la navegación es uno de los campos que más esfuerzos ha concentrado para el ser humano.\\

Pero ¿Cómo es que se puede realizar autonomía en un sistema que debe moverse?, para nosotros es sencillo pues podemos resumirlo en los siguientes pasos:

\begin{itemize}
    \item Ubicar dónde estamos

    \item Posteriormente, tenemos que saber donde está el lugar al que queremos lelgar, para lo cual es indispensable conocer el ambiente o entorno, además de haber memorizado la estructura del mismo.
\end{itemize}

De esta forma, una vez que sabemos nuestra posición y nuestro destino, tenemos que plantear una ruta que una a ambas posiciones. Posteriormente procedemos a caminar para completar la ruta que hemos previsto, sin embargo, durante esa ruta puede que nos salga por ejemplo un perro de casa y ante ello tengamos que esquivarlo para finalmente llegar a nuestro destino.

Este proceso que acabamos de describir y que nos parece tan simple como humanos, ha requerido de bastantes años de investigación.Sin embargo, en los pasos previamente identificados se encuentran los retos de la navegación autónoma que a continuación serán descritos brevemente :

\begin{itemize}
    \item Localización: 
    
    El bloque de localización trata de estimar la posición actual del robot respecto a las coordenadas del mundo. Normalmente, en un robot móvil, se suelen tener tres fuentes de información:
    \begin{itemize}
        \item La posición según la odometría obtenida mediante encoder's
        \item Las observaciones de los sensores: ya sea mediante SLAM's o IMU's
        \item Un mapa del entorno: \\
            Idealmente debería ser creado por un proceso de mapeo o mapping, pero también se puede trabajar con planos elaborados que se ajusten al entorno
        
        
    \end{itemize}

    \item Mapeo: 
    
    El problema del mapeo consiste en construir una representación útil del entorno del robot. En navegación, un mapa se suele utilizar básicamente para dar soporte a otros dos procesos: la localización y la planificación de rutas. 
    
    \item Planificador de rutas: 
    
    Un planificador de trayectorias debe ser capaz de generar una trayectoria libre de obstáculos entre dos posiciones cualquiera del mapa. 
    
    \item Seguimiento de rutas: El robot debe ser capaz de seguir la ruta calculada evitando obstáculos.
    
    Una vez se calcule la trayectoria, necesitamos que el robot se mueva sobre la misma. Para ello hay que diseñar un proceso que conociendo la trayectoria pueda generar velocidades que los motores ejecuten y hagan que el robot se mueva siguiendo la trayectoria planificada.
    
\end{itemize}

\section{Conclusión}

Con todo lo abordado en este breve ensayo sabemos que bloques forman un sistema de navegación, además de sus respectivas tareas dentro del mismo. Como se puede apreciar, hay muchas tareas que a pesar de que nos resulten realmente sencillas de entender e incluso resolver, la realidad es que estas pueden llegar a ser un problema complejo de resolver. \\

La navegación autónoma si bien en el momento en que redacto este ensayo ya resulta incluso como algo parcialmente aplicable en diferentes ambientes comerciales y humanos, aún hay muchos detalles que supondrán un gran reto en los próximos años, como es el caso de las acciones en tiempo real con base a decisiones previamente analizadas o un mapping mucho más completo y real del ambiente en el que se encuentra nuestro robot o entidad móvil. 


\section{Referencias}

1] ROS.org. The AutoRally Platform. https://www.ros.org/news/robots/autonomous-cars/\\

2] The Construct. How to Start with Self-Driving Cars Using ROS. https://www.theconstructsim.com/start-self-driving-cars-using-ros/\\


\section{ANEXO: Propuesta de proyecto final}

Uno de los mayores retos dentro del Taller de Robótica Abierta es poder competir nuevamente en la ROBOCUP  categoría Rescue, para ello algunos aspectos a desarrollar y afinar son principalmente la navegación autónoma mediante paquetes de ROS como es el caso de Gmaping que sirve principalmente para la recreación de ambientes y a su vez del paquete de NavigationStack para la planificación de rutas mediante grafos. Además de algunas otras tareas de deep learning como reconocimiento de imágenes y reconocimiento de voz.
Comparto ligas del taller y repositorio de Github del FINDER(Robot a competir en la ROBOCUP)
\\
1) https://github.com/TRA-UNAM/FinderV3\\
2) http://dcb.fi-c.unam.mx/TallerRobotica/

\end{document}
