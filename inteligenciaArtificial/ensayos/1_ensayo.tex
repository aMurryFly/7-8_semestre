\documentclass[11pt]{article}
\usepackage[utf8]{inputenc}
\usepackage[spanish]{babel}
\usepackage{apacite}
\usepackage[left=2cm,top=2cm,right=2cm,bottom=2cm]{geometry}
\usepackage{natbib}
\usepackage{graphicx}
\usepackage{fancyhdr}

\title{Ensayo 1:\\ Singularidad y Trascendencia}
\author{Murrieta Villegas, Alfonso}



\graphicspath{ {./images/} }
\pagestyle{fancy}
\lhead{Facultad de ingeniería}
\rhead{UNAM}
\rfoot{Inteligencia Artificial}
\begin{document}

\maketitle

\section{Resumen}

La explosión de las redes sociales y la masiva cantidad de datos ha provocado una fuerte inversión y financiamiento de tecnologías dedicadas a facilitar e incluso "interpretar" y pronosticar fenómenos de distintos índoles. Es aquí donde temas como "Inteligencia Artificial" son más que destacables como una "herramientas" con la intención de poder sistematizar y facilitar diversas acciones o tareas humanas que sin duda han superado las capacidades humanas.\\

La realidad es que en las últimas 2 décadas, la importancia o trascendencia que la tecnología ha realizado en la vida humana ha provocado diversas problemáticas o incluso incógnitas que formarán la futura perspectiva social, preguntas tan habituales como ¿Cuáles serán las consecuencias de la AI socialmente hablando? o ¿hasta qué nivel de "inteligencia" la inteligencia artificial llegará?

\section{Singularidad y la Trascendencia }

Hoy en día estamos realmente en un punto donde la perspectiva de la Inteligencia Artificial no está realmente definida incluso por expertos en el área. Y es que temas a veces tan complejos cómo ' el qué es inteligencia', nos lleva a conjeturas e ideas realmente distintas, e incluso factores como la desmedida exageración por parte de la mercadotecnia para la promoción de este tipo de tecnología, sin duda ha causado muchísimas premisas.\\

Sin embargo, hoy en día las aplicaciones de la Inteligencia Artificial desde un punto más comercial e incluso más técnico y aplicable, es realmente extenso e inimaginable. Las altas expectativas de este tipo de tecnologías además de la masiva cantidad de datos, han originado el que distintos expertos pusieran en marcha un sin fin de nuevos algoritmos o formas de sistematizar tareas que para cualquier ser humano supondrían una carga interminable de trabajo.\\

Sin duda, los expertos coinciden en que hoy en día, no existe una definición formal que pueda realmente describir o representar los limites de la Inteligencia Artificial. Aspectos críticos de la A.I. como sus respectivos 'inviernos', además de suponer sesgos y retrasos necesarios para la humanidad, han provocado que haya ciertas 'ramas' o formas de pensar respecto a estos temas. Además la realidad de la inteligencias artificial a través de la historia realmente nos ha enseñado que a lo largo de los años ha cambiado desde la forma de plantearla técnicamente hablando hasta incluso las aplicaciones o tareas en las que está ya inmiscuida.\\ 

Concretamente el hablar de \textbf{trascendencia} y sobre todo de la \textbf{singularidad} en la Inteligencia Artificial, nos origina distintos panoramas, esto sobre todo debido al punto de vista desde el que se quiera abordar y aplicar la Inteligencia Artificial.\\

Desde un punto de vista técnico, la singularidad tecnológica que no solamente se refiere a la particularización en Inteligencia Artificial, supone sobre todo un cambio que además de ser abismal, realmente es real e incluso se está viviendo poco a poco.\\

En la actualidad los distintos sistemas o inteligencias artificiales son capaces de llevar cálculos realmente masivos además de poder procesar una diversidad de algoritmo matemáticos que hace no más de un siglo eran impensables. \\

Desde victorias por parte de la inteligencia artificial en juegos o deportes clásicos como go o ajedrez hasta el poder predecir eventos y fenómenos financieros o matemáticos, la realidad es que a pesar de que tal vez no se tenga una 'conciencia' propia, el computo ha llegado a fronteras realmente nuevas y es que a pesar de que es más que sabido que las computadoras carecen de la habilidad de entender e interpretar los datos que manejan, se pueden programar para que dependiendo del contexto o ámbito a tratar manejen de mejor o peor manera los datos que se les limite para 'aprender.


\section{Conclusión}

Realmente pienso que el desarrollo masivo de la tecnología además de las fuertes inversiones por parte de grandes empresas de software darán en los próximos años resultados e investigaciones realmente llamativas y buenas en la Inteligencia Artificial. \\

A pesar de que hoy en día no haya una computadora con 'conciencia' propia, no significa que en algunos años o décadas no vaya a existir, incluso cediendo a algunas premisas propias y de optimistas en estos temas, podría ser que incluso la perspectiva de lo que sea inteligencia cambie en los años siguientes.\\

Por  último, el escepticismo por gran parte de la población que lamentablemente está mal informada e incluso desinformada respecto a este tipo de temas no solo ha generado una mal idea de la inteligencia artificial sino incluso un rechazo por esta, la realidad es que la inteligencia artificial no deberíamos verla como algo que compita o nos genere problemas como seres humanos, sino como un complemento y herramienta que nos dará un mejor estilo de vida. \\

Sin duda alguna, estamos en una época que ha tenido una gran explosión tecnológica y que poco a poco esta misma inercia nos dará un sin fin de buenos resultados a la humanidad e incluso al mismo conocimiento humano.



\section{Referencias}

1] Jayshree Pandya. The Troubling Trajectory Of Technological Singularity. Recuperado el 10 de  https://www.forbes.com/sites/cognitiveworld/2019/02/10/\\the-troubling-trajectory-of-technological-singularity/\#789850b66711\\

2] Braga, A. (2019). Ai and the singularity: A fallacy or a great opportunity?



\end{document}
